\documentclass[../DCM2_Verslag.tex]{subfiles}
\begin{document}

\section{Script}
\definecolor{codegreen}{rgb}{0,0.6,0}
\definecolor{codegray}{rgb}{0.5,0.5,0.5}
\definecolor{codepurple}{rgb}{0.58,0,0.82}
\definecolor{backcolour}{rgb}{0.95,0.95,0.92}
\begin{lstlisting}[commentstyle=\color{codegreen},
    keywordstyle=\color{magenta},
    numberstyle=\tiny\color{codegray},
    stringstyle=\color{codepurple},
    basicstyle=\ttfamily\footnotesize,
    breakatwhitespace=false,         
    breaklines=true,                 
    captionpos=b,                    
    keepspaces=true,                 
    numbers=left,                    
    numbersep=5pt,                  
    showspaces=false,                
    showstringspaces=false,
    showtabs=false, language=Python]
import serial
import subprocess
import time
import logging
import threading
#WiFi configuration
WiFi_SSID = 'TestAP'
WiFi_Password = "TestPassword"

# Needles for finding esp ip-addr in UART response
ip_search_str = "sta ip:"
ip_search_end_str = ", mask"

# Needles for finding bandwidth numbers in iperf output response
result_begin_str = "Bytes  "
result_end_str = "bits/sec"

# Setup command for initiating WiFi
Setup_CMD = "sta "+WiFi_SSID+" "+ WiFi_Password +"\r"

Initiate_server_CMD = "iperf -s -w "

Package_size_step = 10000

Read_buff_size = 200

def init_esp_iperf_server(s, window_size):
    s.write((Initiate_server_CMD+str(window_size)+"\r").encode())

def run_local_iperf_instance(server_ip_addr, options):
    proc = subprocess.Popen(["iperf -c "+server_ip_addr+" "+options], stdout=subprocess.PIPE, shell=True)
    (output, err) = proc.communicate()
    outputstr = output.decode()
    return outputstr


def find_substring(src_str, begin_str, end_str):
    index = src_str.index(begin_str) + len(begin_str)
    index_stop = src_str.index(end_str)
    return src_str[index:index_stop]
    
def init_esp_wifi(ser):
    ser.write(Setup_CMD.encode())
    print("Initializing ESP-Wifi..", end='')
    read_buffer_str = ""
    while(read_buffer_str.__contains__(", mask") == False):
          read_buffer = ser.read(Read_buff_size)
          read_buffer_str = read_buffer.decode('utf-8')
          print('.', end='')
    time.sleep(2) #Wait for the ESP32 to initialize wifi connection, takes approximately 2 secs
    return find_substring(read_buffer_str, ip_search_str, ip_search_end_str)     
       
def reset_esp32(ser):
    ser.write("restart\r".encode())
    ser.flushOutput()
    time.sleep(5) #Wait for the ESP32 to restart, takes approximately 5 secs
       
def main():
    ser = serial.Serial(
            # Serial Port to read the data from
            port='/dev/ttyUSB0',
 
            #Rate at which the information is shared to the communication channel
            baudrate = 115200,
   
            #Applying Parity Checking (none in this case)
            parity=serial.PARITY_NONE,
 
           # Pattern of Bits to be read
            stopbits=serial.STOPBITS_ONE,
        
            # Total number of bits to be read
            bytesize=serial.EIGHTBITS,
            # Number of serial commands to accept before timing out
            timeout=0
    )
    file1 = open("log.txt", "w")
    for step in range(1,11):
        time.sleep(.1)
        reset_esp32(ser)
        server_ip_addr = init_esp_wifi(ser)
        print(server_ip_addr)
        package_size = Package_size_step*step
        init_esp_iperf_server(ser, package_size)
        outputstr = run_local_iperf_instance(server_ip_addr, " -w "+str(package_size)+" -t 5")
        print(outputstr)
        result = find_substring(outputstr, result_begin_str, result_end_str)
        file1.write(result+"\n")
        file1.write(str(package_size)+"\n")
    file1.close()

        
if __name__=="__main__":
    main()      

\end{lstlisting}
\end{document}
