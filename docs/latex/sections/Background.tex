\documentclass[../DCM2_Verslag.tex]{subfiles}
\begin{document}

Dit hoofdstuk gaat kort in op achtergrond informatie die benodigd is om bepaalde delen van het onderzoek te kunnen begrijpen.

De eerste sectie van dit hoofdstuk geeft de benodigde achtergrond informatie over embedded systemen.

De tweede sectie gaat over tcp/ip stacks

De derde sectie gaat over sockets

\section{Embedded systemen}
Embedded systemen is een woord van de laatste jaren, maar embedded systemen bestaan al veel langer. Het enige wat nodig is, is een blik werpen op de apparaten om je heen: Telefoons, Modems, Televisies en koffie apparaten om een paar voorbeelden te noemen. 

Deze sectie geeft een korte introductie in embedded systemen.

\subsection{Het verschil tussen een computer en embedded systeem}
Een embedded systeem wordt vaak beschreven als een systeem die een bepaald aantal vaste taken moet uitvoeren met beperkte ingebouwde functionaliteit. Dit zijn bijna altijd onzichtbare mini computers (microcomputers) die ingebouwd zitten in apparaten. Maar het kunnen ook chips met programmeerbare hardware zijn (FPGA's of CPLD's). 

Het grote verschil tussen een computer en een embedded systeem zijn dat embedded systemen ook kunnen bestaan uit programmeerbare hardware die in werking heel anders werkt dan een normale computer. Maar ook tussen de minicomputers (microcontrollers) en normale computers zitten grote verschillen. De minicomputers hebben vaak veel minder geheugen en processorkracht dan een normale computer. 

\end{document}
