\documentclass[../DCM2_Verslag.tex]{subfiles}
\begin{document}

Wereldwijd komen er elk jaar steeds meer embedded systemen bij met internet functionaliteit.
Dit brengt naast een hoop mooie mogelijkheden voor de industrie, ook een hoop uitdagingen mee voor fabricanten en ingenieurs. De grootste uitdaging van deze embedded systemen is om een stabiele softwarebasis te schrijven die de hardware assisteert bij het maken en in stand houden van de verbinding. De basis van deze software is een tcp/ip stack. De standaarden en technische eisen van de stack staan vastgelegd, de implementatie echter verschilt. Om de prestaties van een embedded systeem met netwerk functionaliteit vastteleggen zijn uitgebreide testen nodig. 

\section{Doel en motivatie}
Het doel van dit onderzoek is het uitzoeken welke prestaties behaald kunnen worden op een veel voorkomend embedded systeem. Dit onderzoek zal met behulp van Iperf performance testen vastleggen wat de prestaties zijn met verschillende verbindingsparameters.\\\\De resultaten zullen helpen bij het vastleggen van de relatie tussen de verbindingsparameters en bandbreedte. \\Dit bescheven werk zal de nadruk leggen op de werking en het testen van tcp/ip stacks op embedded systemen. De testprocedure voor dit onderzoek kan ook nageproduceerd worden op een generieke computer.\\
De motivatie voor dit werk is de eigen interesse voor computernetwerken en embedded systemen.
\clearpage
\section{Onderzoeksvragen}
Dit onderzoek zal zich richten op het uitzoeken welke prestaties behaald kunnen worden op een ESP32 microcontroller van het merk Espressif. In het onderzoek zullen de hoofvraag en deelvragen beantwoord worden. 

\subsection{Hoofdvraag}
De hoofdvraag voor dit onderzoek is:\\
"What performance can be achieved with the TCP/IP stack on the ESP32?"

\subsection{Deelvragen}
De deelvragen aanvullend op de hoofdvraag zijn:\\
"What is the most efficient package size?"\\
"What is the maximum bit rate?"\\
"What applications can be supported with this system?"\\

\end{document}