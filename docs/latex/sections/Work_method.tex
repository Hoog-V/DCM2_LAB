\documentclass[../DCM2_Verslag.tex]{subfiles}
\begin{document}

Dit hoofdstuk licht de onderzoeksmethode en testopstelling toe. 

\section{ESP32}
Het platform wat onderzocht wordt is de ESP32. De ESP32 is een microcontroller van de Chinese fabrikant Espressif.\\ De microcontroller bevat een WiFi en Bluetooth (met ble) peripheral die met behulp van de open source sdk aangestuurd kan worden. De ESP32 is een veelgebruikte microcontroller in de hobbyscene. Dit komt voornamelijk door de goede ondersteuning voor de arduino sdk en de ingebouwde WiFi peripheral.
\\\\
Andere redenenen voor de populariteit van deze microcontroller zijn:\\
- De twee xTensa lx6 kernen\\
- 4 MB flash (uitbreidbaar tot 16MB met een SPI flash chip)\\
- Freertos ondersteuning\\\\
Dit onderzoek gaat zich richten op de WiFi peripheral met het Iperf voorbeeld uit de SDK. Dit voorbeeld heeft een vrijwel complete implementatie van Iperf v2. 
\section{Testopstelling}


\end{document}
